% [[- Chapter name
\chapter*{Introduction: Ethics as Practical Science}
% -]] Chapter name

% [[- Summary
\section*{Summary}
Ronald Polansky, the editor of the book, offers a rapid overview of
Aristotle's \NE\ as well as praise for its continuing importance and value.
% -]] Summary

% [[- Introduction
\section*{Introduction}

Aristotle's \NE\ is essential but difficult.  Aristotle wrote perhaps the first
systematic treatises on ethics, but his though was grounded in the
investigations of Socrates and Plato.  As Kantian and Utilitarian ethics have
proven to be less appealing and applied ethics and virtue ethics increasingly
of interest, Aristotle's importance has only grown. However, the \NE\ is a very
difficult work and requires a great deal of exegesis and commentary.

Aristotle's  primary focus is ``the best sort of life to lead", and he
considers such important topics as ``happiness, the role of chance or fortune
in life, the place of character and intellect, deliberation and choice, the
contrast of making and doing, desire overriding our better judgment and the
importance of friendship and pleasure" \citep[1]{polansky2014a}.
% -]] Introduction

% [[- Eudaemonism
\section*{Eudaemonism}

Ancient ethics is eudaemonist.  It seeks the ultimate goal of human life.
Systematic eudaemonism begins with Socrates and continues in Plato's dialogues.  Such a view of ethics continued until the sixteenth century CE.  Ancient
eudaemonists agreed that the goal was εὐδαιμονία (roughly \textit{happiness}),
but they disagreed about what exactly εὐδαιμονία was.

Aristotle builds upon the work that Socrates and Plato had done and adds to
their achievements.  In particular, Aristotle's approach is so systematic that
it deserves to be called `scientific'.  In Aristotle's mind, ethics is indeed
a science, but a practical one as opposed to a theoretical or productive
science.\footnote{The details of Aristotle's division of the sciences into
theoretical, practical, and productive are not essential here, but see
\citet{polansky2014a} 3--4.}
% -]] Eudaemonism

% [[- Aristotle and cultural limitations
\section*{Aristotle and cultural limitations}

Polansky responds to criticism from Bernard Williams and Alasdair
MacIntyre that Aristotle's practical science is limited to its culture of
origin (at best) and that it relies on beliefs we can no longer
hold.\footnote{See \citet{williams1985} and \citet{macintyre2007}.}

Polansky replies with a two-pronged argument.  First, he points to the
universality in Aristotle's approach to practical science.  Aristotle
begins with the human animal as such, and in that sense his ethics apply to
\emph{all} humans, not merely Greeks.  Second, modern morality is far more
narrow than ancient eudaemonist ethics.  Modern practical ethics is
concerned almost exclusively with right action---and a narrow view even of
that.  By contrast, Aristotle has a much larger canvas, and thus
Aristotelian ethics is larger not narrower than modern morality.
% -]] Aristotle and cultural limitations

% [[- Practical versus theoretical wisdom
\section*{Practical versus theoretical wisdom}

Polansky argues on behalf of Aristotle's ranking of various parts of the
good life, namely virtue, practical wisdom, and contemplation.  According
to Polansky, Aristotle believes that contemplation or theoretical wisdom
trumps practical wisdom.  Polansky acknowledges that a life of practical
wisdom and natural virtue is \emph{a} good life, but he denies that
Aristotle believes it to be the best life for people.  Instead, the life of
theoretical wisdom or contemplation is a better one because it best
fulfills all the criteria for a good life: it is most continuous, pleasant,
self-sufficient, loved for itself and leisurely.%
\footnote{\citet[12]{polansky2014a} citing \NE\ X 7.}
% -]] Practical versus theoretical wisdom

% [[- Conclusion
\section*{Conclusion}

Polansky concludes by suggesting that Aristotle has a great deal to teach
us and by pointing out that the essays that follow often disagree with each
other and with Polansky on various points of interpretation, both large and
small.  These disagreements themselves are, according to Polansky, signs of
the continuing vitality of Aristotle's \NE.
% -]] Conclusion
