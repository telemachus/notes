% [[- LaTeX prelude
\documentclass[12pt,letterpaper]{article}

\usepackage[no-math]{fontspec}
\setmainfont{Baskerville}

\usepackage[nolocalmarks]{polyglossia}
\setdefaultlanguage{english}
\setotherlanguage[variant=classic]{latin}
\setotherlanguage[variant=ancient]{greek}
\newfontfamily\greekfont[Script=Greek,Scale=MatchLowercase]{GFS Neohellenic}

\usepackage{fnpct}

\usepackage{titlesec}
\titleformat*{\section}{\large\bfseries}
\titleformat*{\subsection}{\bfseries}
\titleformat*{\subsubsection}{\bfseries}

\usepackage{parskip}
\usepackage{csquotes}
\DeclareAutoPunct{.,}
\renewcommand{\mkcitation}[1]{\footnote{#1}}
\renewcommand{\mktextquote}[6]{#1#2#4#5#3#6}

\usepackage[style=windycity,citetracker=context,backend=biber]{biblatex}
\addbibresource{kant.bib}

\usepackage{enumitem}
\setlist{noitemsep}
\usepackage[super]{nth}

\begin{hyphenrules}{latin}
    \hyphenation{}
\end{hyphenrules}

\begin{hyphenrules}{greek}
    \hyphenation{}
\end{hyphenrules}

\usepackage{fancyhdr}
\fancypagestyle{notes}{%
    \fancyhf{}
    \renewcommand{\headrulewidth}{0pt}
    \lhead{}
    \chead{\MakeUppercase{Class Readings on Kant's \textit{Critique of Pure Reason}}}
    \rhead{}
    \lfoot{}
    \cfoot{\thepage}
    \rfoot{}
}
\fancypagestyle{references}{%
    \fancyhf{}
    \renewcommand{\headrulewidth}{0pt}
    \lhead{}
    \chead{\MakeUppercase{References}}
    \rhead{}
    \lfoot{}
    \cfoot{\thepage}
    \rfoot{}
}

\newcommand{\MONTH}{%
  \ifcase\the\month
  \or January% 1
  \or February% 2
  \or March% 3
  \or April% 4
  \or May% 5
  \or June% 6
  \or July% 7
  \or August% 8
  \or September% 9
  \or October% 10
  \or November% 11
  \or December% 12
  \fi}
% -]] Latex prelude

% [[- LaTeX document
\begin{document}

% [[- Title page
\begin{titlepage}
\title{Notes on Kant's \textit{Critique of Pure Reason}}
\author{Peter Aronoff}
\date{January 2021--\MONTH\ \the\year}
\maketitle
\thispagestyle{empty}
\end{titlepage}
% -]] Title page

\pagestyle{notes}

% [[- Metaphysics and the ``Fiery Test of Critique''
\section*{Metaphysics and the ``Fiery Test of Critique''}


% -]] Metaphysics and the ``Fiery Test of Critique''

% [[- Rational Metaphysics: The Highest Aims of Speculative Reason
\subsection*{Rational Metaphysics: The Highest Aims of Speculative Reason}

In this chapter, James O'Shea introduces readers to Kant's \textit{Critique of Pure Reason}.
O'Shea sets out Kant's goals and explains how Kant saw himself in the history of metaphysics.
According to O'Shea, Kant ``intended both to tear down and to rebuild\ldots rational metapysics'' (13).%
\footcite[Unless I say otherwise, all references to O'Shea come from][]{kants-critique-pure-reason-oshea-2014}
In the first two chapters of his book, O'Shea focuses on what Kant attacks.
Later, he will discuss Kant's own views.
% -]] Rational Metaphysics: The Highest Aims of Speculative Reason

% [[- What is metaphysics?
\subsubsection*{What is metaphysics?}

O'Shea uses Kant's \textit{Lectures on Metaphysics} to explain how Kant understood metaphysics.
First, it's worth saying that Kant had a positive view of the importance of metaphysics in his early lectures.
Kant paints a traditional picture of metaphysics.
According to Kant, Aristotle creates metaphysics as we know it.
On the other hand, and this is me speaking not O'Shea, when Kant describes metaphysics, he seems to me to have Descartes rather than Aristotle in mind.
According to Kant, when we study metaphysics we study nature insofar as we can think about nature \textit{a priori} and without sensory experience.
When we study nature in this way, we focus on three things above all: (1) god, as the beginning of all things; (2) freedom, by which Kant seems to mean our ability to follow reason despite our physical limitations and desires, and (3) immortality, i.e., whether the soul will survive the death of the body.
Aristotle does not focus on these three things at all, but Descartes does.

Regardless of who Kant follows, he believes that prior metaphysics has utterly failed to prove that god exists, we are free, and souls are immortal.
Even worse, Kant believes that metaphysics is bound to fail.
Kant says in Preface-A that (i) people will inevitibly ask metaphysical questions which (ii) they will inevitably fail to answer.
And yet, Kant also thinks that, in some sense, he has avoided this failure.
Presumably, this involves his \textit{Copernican revolution}: Kant succeeds by changing the terms of the debate.
Thus, he answers questions that are related to the original questions but which we can answer.
% -]] What is metaphysics?

% [[- Theoretical reason versus practical reason
\subsubsection*{Theoretical reason versus practical reason}

According to O'Shea, Kant distinguishes three types of reason: theoretical, speculative, and practical reason.
Theoretical reason considers how things necessarily are.
Practical reason considers how things should be.
Sometimes Kant uses ``speculative reason'' as a synonym for ``theoretical reason,'' but in more precise moods, he distinguishes them.
Kant believes that theoretical reason cannot establish that god or free will exist or that souls are immortal; from the perspective of theoretical reason, these questions have no answer.
Sometimes Kant uses ``speculative reason'' to describe attempts to answer such questions despite their being unanswerable.
(In these places, ``speculative'' is pejorative, I suppose.)
From the perspective of practical reason, however, Kant thinks that he can establish that god exists, that we enjoy free will, and that we are immortal.
Thus, speculative reason attempts to answer questions in a way that they cannot be answered instead of switching from a theoretical to a practical point of view.
(Kant argues that, e.g., free will exists using an argument that sounds to me somewhat like Strawson in ``Freedom and Resentment.'')
% -]] Theoretical reason versus practical reason

% [[- Denying speculative knowledge, making room for rational faith
\subsubsection*{Denying speculative knowledge, making room for rational faith}

Kant feared that metaphysics made peole less likely to believe in god, free will, and immortality, but he also believed that he turn things around.
He denied that theoretical reason could prove that god and free will existed and that souls were immortal in order that he could then show that practical reason \textit{could} establish these same things.
As Kant says in Preface-B, he believed that is necessary ``to deny knowledge in order to make room for faith'' (Bxxx).
However, as O'Shea notes, Kant does not deny knowledge generally or absolutely here; he only denies that one type of knowledge (theoretical reason) can demonstrate some results (that god and free will exist and that souls are immortal) (18--19).
Kant still believes that theoretical knowledge can demonstrate many other things.
% -]] Denying speculative knowledge, making room for rational faith

% [[- \textit{a priori} versus \textit{a posteriori}, pure versus empirical
\subsubsection*{\textit{a priori} versus \textit{a posteriori}, pure versus empirical}

Kant distinguishes between \textit{a priori} and \textit{a posteriori} knowledge or thought.
\textit{A priori} thought is universal, necessary, and independent of all experience.
Universal and necessary are relatively clear.
In the case of, e.g., causality, we may say that every event must have a cause.%
\footnote{O'Shea uses this example on 21.}
But it can be a bit more difficult to explain how any thought is independent of all experience.

Insofar as I understand O'Shea here, he argues the following.
Every person will have some experiences that lead to their knowing, say, that two plus two equals four.
But the knowledge that two plus two equals four does not ultimately depend on those experiences.
We can abstract away the abstract knowledge from whatever particulars happened to get us to the knowledge.
The proof that two plus two equals four does not rely on anything like the apples or coins that someone used to teach you that two plus two equals four.

By contrast, \textit{a posteriori} knowledge is empirical.
We cannot separate \textit{a posteriori} knowledge from experience because experience is required to justify or proove the knowledge.
Examples of \textit{a posteriori} knowledge are ``8th Street in NYC is south of 42nd Street'' and ``water is H\textsubscript{2}O.''

Kant further distinguishes between pure and impure \textit{a priori} knowledge.
(To make things more complicated, Kant sometimes uses \textit{mixed} instead of \textit{impure}.)
A pure \textit{a priori} thought has no empirical content whatsoever.
But an impure (or mixed) \textit{a priori} thought has some emprical content, even if the thought as a whole is knowlable independently of experience (of particular objects?).
% -]] \textit{a priori} versus \textit{a posteriori}, pure versus empirical

% [[- `Appearances' Versus `Things in Themselves': Kant's Transcendental Idealism
\subsection*{`Appearances' Versus `Things in Themselves': Kant's Transcendental Idealism}

% -]] `Appearances' versus `things in themselves': Kant's transcendental idealism

% [[- Sensible `appearances' versus intelligible `things in themselves'
\subsubsection*{Sensible `appearances' versus intelligible `things in themselves'}

Although there are some differences between them, Kant and critics of Kant generally treat \textit{appearances} and \textit{phenomena} as the same and \textit{things in themselves} and \textit{noumena} as the same.
The word \textit{appearances} here is not pejorative, and it does not imply error.
Appearances are simply ordinary objects in space and time that our senses can (and do!) perceive.
In this sense, the appearance or phenomenon is the thing we perceive, not our act of perceiving it.
Kant includes as appearances things we can perceive indirectly even if the object itself is not directly observable.
% -]] Sensible `appearances' versus intelligible `things in themselves'

% [[- Transcendental idealism
\subsubsection*{Transcendental idealism}

O'Shea asks: Why does Kant use `appearance' instead of `physical object' and `mere appearances to us' instead of `psychological states' (28)?
% -]] Transcendental idealism

% [[- Bibliography
\newpage
\pagestyle{references}
\defbibfilter{primary}{%
    keyword=primary
}
\defbibfilter{secondary}{%
    keyword=secondary
}

%\nocite{*}
\printbibliography[filter=primary,title={Primary Sources}]
\printbibliography[filter=secondary,title={Secondary Sources}]
% -]] Bibliography

\end{document}
% -]] LaTeX document
