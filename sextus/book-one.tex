% [[- Chapter title
\chapter{Book I}
% -]] Chapter title

% [[- The most general difference of philosophies (1--4)
\section{The most general difference of philosophies (1--4)}


Sextus begins thus:

\begin{quote}
    When people inquire into any matter, the likely result is either (i) a discovery, or (ii) a denial of discovery and an agreement of inapprehensibility, or (iii) a continuation of inquiry.
\end{quote}

In other words, if someone investigates, say, if \textit{p}, then we can expect them to find out, to fail to find out and say that it is not possible to find out, or to keep investigating.

Sextus uses this to distinguish between three types of philosopher. Those properly called dogmatists think that they have discovered the truth. Academics assert that it is not possible to discover the truth. They are therefore a kind of negative dogmatist. The skeptics continue to search for answers.

Sextus will speak about the skeptics, and even there only with provisos. He disavows any attempt to affirm strongly how things actually are. Instead, he will ``report like a storyteller about each matter in accordance with how things current seem to him.'' \textbf{Nota bene}: I wonder what we would find if we investigated in detail how other writers use the adverb \textgreek{ἱστορικῶς}, which Sextus uses to qualify \textgreek{ἀπαγγέλλομεν} here.
% -]] The most general difference of philosophies (1--4)

% [[- The accounts of skepticism (5--6)
\section{The accounts of skepticism (5--6)}


Sextus subdivides skeptical philosophy into a general and a specific part. He explains that the general part of skepticism concerns the character of skepticism itself. In the general part, we consider questions about the intention, starting points, criterion, goals of skepticism, skeptical assertions, and the differences between skepticism and nearby schools of thought. In the specific part, the skeptic argues against each part of what other people call philosophy. I think this alludes to the division of philosophy into logic, physics, and ethics.
% -]] The accounts of skepticism (5--6)

% [[- The names applied to skepticism (7)
\section{The names applied to skepticism (7)}


Sextus considers several names that people use for \textgreek{ἡ σκεπτικὰ ἀγωγή}: zetetic, ephetic, aporetic, and Pyrrhonian. People call skeptics zetetic because they inquire and investigate so conspicuously. People call skeptics ephetic because of the experience they undergo after inquiry. That is, people call skeptics ephetic because they withold judgment or suspend judgment. People call them aporetic either because they are at a loss about everything and inquire or because they they are at a loss about assent and denial. That is, Sextus appears to distinguish here between the beginning and process of skeptical inquiry, on the one hand, and the failure to conclude this inquiry, on the other hand. Finally, people call skeptics Pyrrhonian because ``Pyrrho appears to us to have advanced in skepticism more bodily and conspicuously than anyone before him.''

Scholars have taken Sextus to be cautious or guarded in his comments about Pyrrho, but I am not so sure. Sextus certainly doesn't call Pyrrho the first skeptic, but I think that what he \textit{does} say is perfectly full throated.

However, I am unsure how to interpret \textgreek{σωματικώτερον} here. LSJ doesn't have a lot on the adverbial use of this adjective. They list a literally use (``corporeally''), a use in astrology where it is opposed to ``in aspect'' (huh?), and a legal context where punishment can be taken ``in money or physically.'' I suppose that the sense is that Pyrrho committed himself to skepticism ``body and soul,'' as we might say.
% -]] The names applied to skepticism (7)

% [[- What is skepticism? (8--10)
\section{What is skepticism? (8--10)}

Sextus tells us that skepticism is something one does rather than a body of ideas. This only makes sense since a Pyrrhonian skeptic has no characteristic body of doctrine. (See Frede and Striker on this.) More specifically, Sextus calls skepticism ``an ability to place appearances and thoughts in conflict in any way at all, from which ability skeptics arrive first at suspension of judgment and then at ataraxia as a result of the equal strength of opposed matters and arguments.''

Sextus then goes through each part of this definition giving us further guidance on how to take his words. In many cases, he specifies how to take his words. For example, by \textgreek{δύναμις} he means simply ability, and by \textgreek{φαινόμενα} he means perceptual appearances since he opposes them (here) to \textgreek{νοητά}. However, in other cases, Sextus offers more than one interpretation. For example, he suggests that we can join \textgreek{καθ᾽οἱονδήποτε τρόπον} with \textgreek{δύναμις} or with \textgreek{ἀντιθετικὴ φαινομένων τε καὶ νοουμένων}. I find the effect of this jarring, but it is a feature of Sextus's writing. He often tells us that things are ambiguous, and he does not much care to settle those ambiguities.

Sextus offers a third interpretation of \textgreek{καθ᾽οἱονδήποτε τρόπον} which I find most interesting of all. We can take this phrase closely with \textgreek{φαινομένων τε καὶ νοουμένων} insofar as Sextus does not care \textit{how} things that appear or are thought appear or are thought. As he says ``we simply take these.'' The point seems to be that, for the moment at least, Sextus is willing to take these as givens which he does not care to investigate.
% -]] What is skepticism? (8--10)

% [[- The skeptic (11)
\section{The skeptic (11)}

Sextus is very brief on this topic: a Pyrrhonian skeptic is pretty much just someone who has the ability outlined in iii.
% -]] The skeptic (11)

% [[- The sources of skepticism (12)
\section{The sources of skepticism (12)}

The causal source of skepticism is the hope of \textgreek{ἀταραξία}. To explain, Sextus tells a story. Very intelligent people were disturbed when they noticed the anomaly in things, and they were at a loss as to which view about things they should assent to. So they began to investigate. Their intention was to decide how things are and thus free themselves from this disturbance. This is the initiating cause of skepticism.

Sextus adds that the cause of the skeptical party or constitution is primarily the claim that there is an equal argument for every argument. Sextus does not say it, but by this he seems to mean that \textgreek{isosthenia} brings the skeptics together as a group. What he does say is that \textgreek{isosthenia} leads them to cease to dogmatize. In a way, this too \textit{is} what defines them as a group. (See i.)
% -]] The sources of skepticism (12)

% [[- Does the skeptic have beliefs? (13--15)
\section{Does the skeptic have beliefs? (13--15)}

Sextus begins by telling us a sense of ``have beliefs'' (\textgreek{δογματίζειν}) in which he does \textit{not} say that the skeptic does not have beliefs. This is odd, and makes things complicated right away. There is a sense in which to have beliefs is simply to agree to some matter. Sextus does not say that skeptics do not do this since, as he says, ``the skeptic assents to experiences which are perceptually forced on them'' (\textgreek{τοῖς γὰρ κατὰ φαντασίαν κατηναγκασμένοις πάθεσι συγκατατίθεται ὁ σκεπτικός}). He then gives two examples: if a skeptic is warmed or cooled, the skeptic will not say ``I am not warmed or cooled.'' Notice that again this is all multiply negative. 

Sextus then considers a way in which the skeptic will not have beliefs. The skeptic will not have beliefs if by ``have beliefs'' we mean ``assent to some unclear matter investigated by the sciences.'' In fact, Sextus continues, the skeptic will not have beliefs in this sense since the skeptic will not assent to \textit{anything} unclear.

According to Sextus, when the skeptic utters the skeptical phrases such as ``no more,'' the skeptic also does not hold beliefs. A person who holds beliefs posits as real (or ``as really existing'' or ``as true,'' \textgreek{ὡς ὑπάρχον}) what they have beliefs about, but the skeptic does not take this attitude towards the skeptical phrases. The skeptical phrases apply to themselves as much as to other things. Hence, the skeptic does not think of them as altogether real or true. Most importantly, the skeptic can utter their special phrases and not hold beliefs because the skeptic only says what appears to them and reports their (passive) experience without belief and without affirming anything about underlying external realities.

Sextus leaves a great deal unsaid here. In no particular order, (i) Sextus does not clarify the relationship between the beliefs that a skeptic may have and ordinary belief, (ii) Sextus does not explicitly state whether a skeptic can have non-dogmatic beliefs about the subject matter of the sciences or philosophy, (iii) Sextus does not say whether skeptical beliefs are limited by subject, strength, formation, or anything else. He lists a few things that a skeptic may not do (``assent to anything unclear'' and ``posit as real'' what they talk about), but this still leaves a great many questions.
% -]] Does the skeptic have beliefs? (13--15)

% [[- Does the skeptic belong to a school? (16--17)
\section{Does the Skeptic Belong to a School? (16--17)}

Whether a skeptic belongs to a school depends on what you mean by ``belong to a school.'' If you mean that someone who belongs to a school accepts many beliefs that (i) are internally coherent, (ii) fit with appearances, and (3) involve assent to unclear matters, then the skeptic does not belong to a school. However, the skeptic does belong to a school in the sense that skeptics have (i) a way of life that (ii) follows some rationale (iii) in accordance with appearances; and (iv) that rationale indicates how it is possible to live in an apparently correct manner. Sextus clarifies that by ``a correct manner'' he means something more simple than ``in accordance with virtue.'' And whatever he means includes the possibility of suspension of judgment. (Sextus adds that the rationale the skeptics follow points to a life in accordance with traditional customs and laws as well as individual feelings.)
% -]] Does the skeptic belong to a school? (16--17)

% [[- Does the skeptic study nature? (18)
\section{Does the skeptic study nature? (18)}

Sextus explains that the skeptic does not study nature in order to make pronouncements about the reality of nature ``with firm conviction.'' Instead, the skeptic studies nature in order to balance arguments against one another and reach \textgreek{ἀταραξία}. Sextus adds that the same applies to ``the logical and ethical parts of what is called philosophy.''
% -]] Does the skeptic study nature? (18)

% [[- Do the skeptics do away with appearances? (19--20)
\section{Do the skeptics do away with appearances? (19--20)}

No, says Sextus, the skeptics do not do away with appearances. First, he reminds us that skeptics do not reject impulses which lead us involuntarily towards assent in accordance with passive perception, as he explained in §17. But, he continues, this just \textit{is} appearance. Next, Sextus argues that when skeptics invetigate if something is actually the way it appears, they are not investigating the appearance. They are investigating the underlying reality (\textgreek{τὸ ὑποκείμενον}). In fact, skeptics grant the appearance as a starting point of investigation. Sextus offers the following example: skeptics might investigate honey's sweetness. They do not question whether the honey \textit{appears} sweet; they question whether it \textit{is} sweet. But they grant without argument that honey appears sweet, and here Sextus uses one of the odd Cyrenaic-like phrases: ``for we are perceptually sweetened'' (\textgreek{γλυκαζόμεθα γὰρ αἰσθητικῶς}). On the other hand, Sextus continues, skeptics investigate whether honey is sweet ``as far as the argument (or account?)'' (\textgreek{ὅσον ἐπὶ τῷ λόγῳ}). Thirdly, Sextus acknowledges that sometimes skeptics do make arguments directly against appearances, but they do so only to make an \textit{a fortiori} argument against the dogmatists. If there is so much debate even about appearances, then we can be even less confident about unclear (i.e., not apparent) matters. Sextus specificies that this argument is meant to counter dogmatic ``rashness'' (\textgreek{προπέτεια, προπετεύεσθαι}). Compare what Descartes says about \textit{précipitation} in \textit{Discours de la méthode, deuxième partie}.\autocite[][69, especially Gilson's note 1]{gilson2016}
% -]] Do the skeptics do away with appearances? (19--20)

% [[- On the criterion of skepticism (21--24)
\section{On the criterion of skepticism (21--24)}

Sextus distinguishes between two senses of ``criterion.'' On the one hand, a criterion (of truth) concerns the reality or unreality of things. On the other hand, a criterion (of action) concerns how one acts, namely what one does and what one avoids. Sextus will discuss the criterion of truth later, in order to refute it. Now, he discusses the skeptic's criterion of action.

According to Sextus, skeptical beliefs about the criterion also show that the skeptics ``cleave to appearances'' (\textgreek{τοῖς φαινομένοις προσέχομεν}). He explains this as follows. The criterion of skepticism \textit{is} ``the apparent,'' (\textgreek{τὸ φαινόμενον}) by which, Sextus adds, he means ``appearance'' (\textgreek{φαντασία}). This appearance is a ``passive experience and unwilled affection'' (\textgreek{ἐν πείσει καὶ ἀβουλήτῳ πάθει}), and it ``is not subject to inquiry'' (\textgreek{ἀζήτητος}). Therefore, nobody investigates whether things appear this way or that; people investigate whether things are as they appear. The appearance itself remains taken for granted and (thus?) not subject to inquiry.

Holding fast to these appearances, skeptics live a (relatively) normal life without belief. Sextus divides (logically) this normal life into four parts. First, there is ``guidance of nature'' (\textgreek{ὑφήγησις φύσεως}); second, ``necessity of feelings'' (\textgreek{ἀνήγκη παθῶν}); third, ``tradition of laws and customs'' (\textgreek{παράδοσις νόμων τε καὶ ἐθῶν}); fourth, ``teaching of skills'' (\textgreek{διδασκαλία τεχνῶν}).

The last three are relatively clear. Hunger and thirst guide us towards food and drink. This is what Sextus means by the ``necessity of feelings.'' Sextus refers to religious observance, or failure of observance, as an example of ``tradition of customs and laws.'' Sextus gives no example for the ``teaching of skills,'' but he clearly has in mind things like medicine, painting, or baking. These are limited domains of knowledge (in an everyday sense) that people master in order to have jobs.

It is harder to say, however, what Sextus means by ``the guidance of nature.'' Here's what Sextus says:
\begin{quote}
    \textgreek{ἐν ὑφηγήσει φύσεως}\dots \textgreek{καθ᾽ἥν φυσικῶς αἰσθητικοὶ καὶ νοητικοί ἐσμεν}

    in the guidance of nature\dots by means of which we are naturally able to perceive and think
\end{quote}
Michael Frede takes this to mean that Pyrrhonists believe ``that human beings, by nature, do have reason and that there is a natural use for it.''\autocite[][95]{frede1988} Gisela Striker, however, argues that we should not ``conclude that this means reasoning'' but rather ``\textgreek{νοητικοί} should probably be taken only in the minimal sense of being able to think and understand language.''\autocite[][119--120, note 7]{striker2001} Striker also refers to II.10. From the start of Book II, Sextus reports and responds to dogmatic arguments that skeptics cannot engage with dogmatists because they do not apprehend (\textgreek{καταλαμβάνειν}) dogmatic views. This argument is something like Meno's paradox, and if dogmatists used it in the way Sextus describes, they were nakedly sophistic. In any case, in II.10 Sextus argues for a sense of \textgreek{νόησις} and \textgreek{νοέω} weak enough that a skeptic can understand the arguments at issue without grasping them as certain truths. At the moment, I'm inclined to agree with Striker and to take the passage in a minimal sense.
% -]] On the criterion of skepticism (21--24)

% [[- The goal of skepticism (25--30)
\section{The goal of skepticism (25--30)}

Before he tells us about the goal (\textgreek{τὸ τέλος}) of skepticism, Sextus briefly explains how he takes ``goal.'' He defines goal here as ``that for the sake of which everything is done or considered, while it itself is not done for the sake of anything, or the furthest object of desire.'' These definitions are traditional, but I wish he had said more about how he thinks that a skeptic can have a goal.\footnote{Note to self: check whether he talks more about goals in the ethical part of \textit{PH} III or \textit{M} XI. A quick glance at both suggests that he does not. Sextus seems, therefore, to take the concept of a goal of life for granted, or, at least, he chooses not to fight the dogmatists on that front.}

Sextus defines the skeptic's goal as ``tranquility in matters of belief'' (\textgreek{ἐν τοῖς κατὰ δόξαν ἀταραξίαν}) and ``moderate feeling in cases that are forced'' (\textgreek{ἐκ τοῖς καηναγκασμένοις μετριοπάθειαν}). At the end of this section, Sextus notes that ``some well-known skeptics add suspension of judgment in investigations'' (\textgreek{τὴν ἐν ταῖς ζητήσεσιν ἐποχήν}). The two-part division reminds me of Epicurus, who defines the \textgreek{τέλος} as \textgreek{ἀταραξία} and \textgreek{ἀπονία}. Both Sextus and Epicurus provide a mental and a physical side of the \textgreek{τέλος}). In addition, both focus more on the mental than on the physical side of things.

Sextus explains \textit{ataraxia} as a happy accident. He tells the story (again) that the skeptics did not expect or hope to find peace of mind in the way that they did. Initially, they sought answers and they wanted to determine which views were true and which were false. Skeptics thought that this was the path to peace of mind. Completely by accident, they fell upon disagreement with equal arguments on either side. As such, they suspended judgment, and then---boom---they achieved peace of mind unexpectedly. In this context, Sextus also tells the story of Apelles trying to paint foam on a horse's mouth to make the same point.

Sextus also claims that dogmatists cannot achieve peace of mind: their belief in real goods and evils precludes them from this happy state. So long as you believe something is really good, you will be disturbed. If you have the good, you will act recklessly because of your immoderate view of hte items importance and you will constantly fear losing it. If you lack the good, you will pursue it tirelessly and agonize that you don't have it. The same problems would occur, \textit{mutatis mutandis}, for what you consider bad.

Sextus considers \textit{metriopatheia} as a matter of doing the best you can with what you cannot change. Sextus admits that skeptics cannot guarantee that they do not suffer: they are susceptible to the world around them. He mentions cold and thirst as examples. However, Sextus claims that non-skeptics\footnote{Sextus refers to these non-skeptics as ``lay people, ordinary people'' \textgreek{ἰδιῶται}. I wonder why he doesn't apply this same criticism to dogmatists. And, on the other hand, I wonder why he doesn't mention ordinary people when he talks about peace of mind.} suffer \textit{twice} in such cases. They experience whatever physical harm is inevitable, but non-skeptics also suffer through mental distress, both in anticipation and when the harm is present.

Let me try to clarify what Sextus does and does not argue in this section. He argues for the following: (i) skeptics can achieve peace of mind, (ii) dogmatists cannot achieve peace of mind, (iii) ordinary people cannot practice moderation of feelings they way that skeptics can. However, Sextus never argues that peace of mind or moderation of feelings are good goals, nor does he argue for their superiority to dogmatic or commonsense goals. This suggests that his arguments are \textit{ad hominem}, but I'm not sure who, exactly, he is arguing with nor whose arguments he borrows here.
% -]] The goal of skepticism (25--30)
