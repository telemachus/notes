% [[- LaTeX prelude
\documentclass[12pt,letterpaper]{article}

\usepackage[no-math]{fontspec}
\setmainfont{Baskerville}

\usepackage[nolocalmarks]{polyglossia}
\setdefaultlanguage{english}
\setotherlanguage[variant=classic]{latin}
\setotherlanguage[variant=ancient]{greek}
\newfontfamily\greekfont[Script=Greek,Scale=MatchLowercase]{GFS Neohellenic}

\usepackage{fnpct}

\usepackage{titlesec}
\titleformat*{\section}{\large\bfseries}
\titleformat*{\subsection}{\bfseries}
\titleformat*{\subsubsection}{\bfseries}

\usepackage{parskip}
\usepackage{csquotes}
\DeclareAutoPunct{.,}
\renewcommand{\mkcitation}[1]{\footnote{#1}}
\renewcommand{\mktextquote}[6]{#1#2#4#5#3#6}

\usepackage[style=windycity,citetracker=context,backend=biber]{biblatex}
\addbibresource{kant.bib}

\usepackage{enumitem}
\setlist{noitemsep}
\usepackage[super]{nth}

\begin{hyphenrules}{latin}
    \hyphenation{}
\end{hyphenrules}

\begin{hyphenrules}{greek}
    \hyphenation{}
\end{hyphenrules}

\usepackage{fancyhdr}
\fancypagestyle{notes}{%
    \fancyhf{}
    \renewcommand{\headrulewidth}{0pt}
    \lhead{}
    \chead{\MakeUppercase{Notes on Kant's \textit{Critique of Pure Reason}}}
    \rhead{}
    \lfoot{}
    \cfoot{\thepage}
    \rfoot{}
}
\fancypagestyle{references}{%
    \fancyhf{}
    \renewcommand{\headrulewidth}{0pt}
    \lhead{}
    \chead{\MakeUppercase{References}}
    \rhead{}
    \lfoot{}
    \cfoot{\thepage}
    \rfoot{}
}

\newcommand{\MONTH}{%
  \ifcase\the\month
  \or January% 1
  \or February% 2
  \or March% 3
  \or April% 4
  \or May% 5
  \or June% 6
  \or July% 7
  \or August% 8
  \or September% 9
  \or October% 10
  \or November% 11
  \or December% 12
  \fi}
% -]] Latex prelude

% [[- LaTeX document
\begin{document}

% [[- Title page
\begin{titlepage}
\title{Notes on Kant's \textit{Critique of Pure Reason}}
\author{Peter Aronoff}
\date{January 2021--\MONTH\ \the\year}
\maketitle
\thispagestyle{empty}
\end{titlepage}
% -]]

\pagestyle{notes}

% [[- Preface A
\section*{Preface A}

Kant starts his first preface sounding like a New Mysterian.
Human reason inevitably asks questions that it cannot answer.
In addition, Kant claims that reason almost inevitably makes mistakes in how it tries to answer these questions.
Kant imagines that we begin with principles ``whose use is unavoidable in the course of experience'' (99).%
\footcite[Unless I say otherwise, all references to Kant's first critique come from][]{critique-pure-reason-cambridge-1998}
We run into trouble when we begin to use ``principles that overstep all possible use in experience'' as we attempt to answer the (unaswerable?) questions (99).
Metaphysics is the field where people battle out controversies that arise among those principles not grounded in experience.

Kant provides a miniature history of metaphysics.
First there was dogmatism.
(I take Kant to mean ancient thinkers and schools all the way up until his time, but Guyer and Wood focus only on Kant's contemporaries and near contemporaries.)
Then there was skepticism.
(Again, I take Kant to mean starting from ancient times and running all the way up to his day.)
All the back and forth has led to indifferentism, which Kant explicitly says is relatively recent.
(I take Kant to have Hume in mind here since Hume begins from skeptical arguments, but he also argues that we will inevitably fall back into some form of belief, despite our skeptical arguments.)

However, Kant believes that we cannot remain indifferent.
First, he thinks that ``human nature cannot be indifferent'' about such questions (100).
This is an interesting contrast with Hume, who appears to believe that we cannot remain skeptical.
On the other hand, Hume may agree that we cannot remain indifferent either.
Perhaps Hume also believes that we cannot remain indifferent.
The difference between the two, if that is right, is that Kant believes that we can earn our dogmatism but Hume does not.
That is, Hume believes that skepticism is correct, but that we cannot remain skeptical in practice.

Kant has a second reason to think that we cannot remain indifferent: the spirit of the times.
Kant believes that the sciences are flourishing and that his age exhibits improved thinking.
Therefore, in his mind, it is all the more appropriate that he attempt to solve the puzzles that metaphysics creates.

Kant promises that the account he provides is complete and certain.
These goals echo things you find in Plato, Aristotle, the Stoics, and Descartes, among many others.
Nevertheless, Kant warns readers that the results may not satisfy their original ``dogmatically enthusiastic lust for knowledge'' (101).
Why not? 
Because dogmatism sets goals that are impossible.%
\footnote{As Kant says, nothing could satisfy dogmatism except ``magical powers in which I am not an expert'' (101).}
Some of what Kant says here strikes me as Wittgensteinian.
He solves some problems by showing that the desired solutions were ``delusions'' (101).
% -]] Preface A

% [[- Preface B
\section*{Preface B}


% -]] Preface B

% [[- Bibliography
\newpage
\pagestyle{references}
\defbibfilter{primary}{%
    keyword=primary
}
\defbibfilter{secondary}{%
    keyword=secondary
}

%\nocite{*}
\printbibliography[filter=primary,title={Primary Sources}]
\printbibliography[filter=secondary,title={Secondary Sources}]
% -]] Bibliography

\end{document}
% -]]
