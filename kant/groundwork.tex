% [[- LaTeX prelude
\documentclass[12pt,letterpaper]{article}

\usepackage[no-math]{fontspec}
\setmainfont{Baskerville}

\usepackage[nolocalmarks]{polyglossia}
\setdefaultlanguage{english}
\setotherlanguage[variant=classic]{latin}
\setotherlanguage[variant=ancient]{greek}
\newfontfamily\greekfont[Script=Greek,Scale=MatchLowercase]{GFS Neohellenic}

\usepackage{fnpct}

\usepackage{titlesec}
\titleformat*{\section}{\large\bfseries}
\titleformat*{\subsection}{\bfseries}
\titleformat*{\subsubsection}{\bfseries}

\usepackage{parskip}
\usepackage{csquotes}
\DeclareAutoPunct{.,}
\renewcommand{\mkcitation}[1]{\footnote{#1}}
\renewcommand{\mktextquote}[6]{#1#2#4#5#3#6}

\usepackage[style=windycity,citetracker=context,backend=biber]{biblatex}
\addbibresource{kant.bib}

\usepackage{enumitem}
\setlist{noitemsep}
\usepackage[super]{nth}

\begin{hyphenrules}{latin}
    \hyphenation{}
\end{hyphenrules}

\begin{hyphenrules}{greek}
    \hyphenation{}
\end{hyphenrules}

\usepackage{fancyhdr}
\fancypagestyle{notes}{%
    \fancyhf{}
    \renewcommand{\headrulewidth}{0pt}
    \lhead{}
    \chead{\MakeUppercase{Notes on Kant's \textit{Groundwork}}}
    \rhead{}
    \lfoot{}
    \cfoot{\thepage}
    \rfoot{}
}
\fancypagestyle{references}{%
    \fancyhf{}
    \renewcommand{\headrulewidth}{0pt}
    \lhead{}
    \chead{\MakeUppercase{References}}
    \rhead{}
    \lfoot{}
    \cfoot{\thepage}
    \rfoot{}
}

\newcommand{\MONTH}{%
  \ifcase\the\month
  \or January% 1
  \or February% 2
  \or March% 3
  \or April% 4
  \or May% 5
  \or June% 6
  \or July% 7
  \or August% 8
  \or September% 9
  \or October% 10
  \or November% 11
  \or December% 12
  \fi}
% -]] Latex prelude

% [[- LaTeX document
\begin{document}

% [[- Title page
\begin{titlepage}
\title{Notes on Kant's \textit{Groundwork}}
\author{Peter Aronoff}
\date{December 2020--\MONTH\ \the\year}
\maketitle
\thispagestyle{empty}
\end{titlepage}
% -]]

\pagestyle{notes}

% [[- Gregory Vlastos on Socratic Irony
\section*{Gregory Vlastos on Socratic Irony}

\begin{quote}

    Here we see a new form of irony, unprecedented in Greek literature to my knowledge, which is peculiarly Socratic.
For want of a better name, I shall call it ``complex irony'' to contrast it with the simple ironies I have been dealing with in this chapter heretofore.
In ``simple'' irony what is said just isn't what is meant: taken in its ordinary, commonly understood, sense the statement is simply false.
In ``complex'' irony what is said both is and isn't what is meant: its surface content is meant to be true in one sense, false in another.
Thus when Socrates says [in Xenophon's \textit{Memorabilia}] he is a ``procurer'' he does not, and yet does, mean what he says.
He obviously does not in the common, vulgar, sense of the word.
But nonetheless he does in another sense he gives the word \textit{ad hoc}, making it mean someone ``who makes the procured attractive to those whose company he is to keep'' (4.57) (31).

\end{quote}

Vlastos primarily considers four examples of complex Socratic irony (see 237 for the first three and Chapter 1 for the fourth):

\begin{enumerate}
    \item Socrates disavows, yet avows, knowledge.
    \item Socrates disavows, yet avows, the art of teaching virtue.
    \item Socrates disavows, yet avows, engaging in politics (\textgreek{πράττειν τὰ πολιτικά}).
    \item Socrates loves, yet does not love, young attractive boys.
\end{enumerate}

The fourth example is very different from the first three.
Perhaps for this reason, Vlastos does not discuss it in his long note at the end of the book.

Vlastos distinguishes simple from complex irony.
If a speaker uses simple irony, they communicate (and intend to communicate) what they mean by words that say (more or less) the opposite of what they mean.
When Socrates uses complex irony, he both means and does not mean what he says.
He means what he says if certain key words are taken in a specific way.
He does not mean what he says if those same key words are taken in another way.
Socrates means what he says when we take his words in his special ways.
If we understand the words in everyday senses, then he does not mean what he says.


As an example, let's consider what Vlastos says about Socrates and knowledge.
According to Vlastos, Socrates sincerely disavows knowledge\textsubscript{c}, but he also sincerely avows knowledge\textsubscript{e}.
Knowledge\textsubscript{c} is certain knowledge, and Socrates believes that he can only have such knowledge if he possesses definitions.
Since he does not have definitions for virtue terms, he does not think that he has knowledge\textsubscript{c}.
Knowledge\textsubscript{e} is elenctic knowledge: that is, it is knowledge produced by means of elenctic testing.
Socrates has this, but he is aware that it is only provisional.

Vlastos also argues that complex irony is a test of sorts for interlocutors.

\begin{quote}

    If you are young Alcibiades courted by Socrates you are left to your own devices to decide what to make of his riddling ironies.
If you go wrong and he sees you have gone wrong, he may not lift a finger to dispel your error, far less feel the obligation to know it out of your head.
If this were happening over trivia no great harm would be done.
But what if it concerned the most important matters—whether or not he loves you?
He says he does in that riddling way which leaves you free to take it one way though you are meant to take it in another, and when he sees you have gone wrong he lets it go.
What would you say?
Not, surely, that he does not care that you should know the truth, but that he cares more for something else: that if you are to come to the truth, it must be by yourself for yourself (44).

\end{quote}

What do I like about this account?
First, I agree that Socrates sometimes means what he says, in one sense, while not meaning what he says, in another sense.
Second, I agree that Socrates pushes interlocutors to discover things for themselves rather than simply telling them things.
(Nevertheless, I wonder if Vlastos underestimates how much help Socrates gives.
I think we can argue that he does various things to tip his hand and push the interlocutors towards insight, even if he never completely gives away the game.)
Third, I think that I agree broadly with Vlastos about the attitude Socrates takes towards teaching.
Finally, I may also agree broadly with Vlastos about Socrates and politics.

What do I not like about this account?
First, I still see this as deceptive speech.
Vlastos seems to admit as much when he says that Alcibiades must figure things out on his own.
Socrates will not and does not give him answers.
Second, I do not agree with the account Vlastos gives of Socratic irony about knowledge.
To begin with, I do not believe that the elenchus can produce anything Socrates would call knowledge.
In addition, it is not clear to me that Socrates has core principles that he has derived from the elenchus itself.
(He has core principles, but these seem to me to be \textit{a priori} for him—or at least untested.)
Third, I do not see why it helps to call this ``irony'' at all.
That said, I am beginning to think that I want to limit the term ``irony'' too much.
Clearly, the word has spread.
I should maybe accept that.
% -]] Gregory Vlastos on Socratic Irony

% [[- Bibliography
\newpage
\pagestyle{references}
\defbibfilter{primary}{%
    keyword=primary
}
\defbibfilter{secondary}{%
    keyword=secondary
}

\nocite{*}
\printbibliography[filter=primary,title={Primary Sources}]
\printbibliography[filter=secondary,title={Secondary Sources}]
% -]] Bibliography

\end{document}
% -]]
