\hyphenation{συζ-η-τῆ-σαι}

% [[- Recollection in Meno
\section{Recollection in \book{Meno}}

% [[- How did we get here?
\subsection{How did we get here?}

Before considering Meno's challenge, I want to review, at least briefly, what motivates it. At the start of the dialogue, Meno asks Socrates how {\g ἀρετή} is acquired: through teaching, practice, or nature (70A1--3)? Socrates says that he cannot answer because he cannot even say what {\g ἀρετή} is. This leads to an examination of whether Meno can define {\g ἀρετή}, and that in turn results in {\g ἀπορία} (70A--80D3). Socrates invites Meno to continue to inquire together with him ({\g σκέψασθαι, συζητῆσαι} 80D4). At this point Meno responds with his challenge concerning inquiry.

The way that Socrates handles Meno is familiar from several early dialogues. The key elements are (i) Socrates denies knowledge, (ii) Socrates invites an interlocutor to explain something, (iii) Socrates demolishes the intelocutor's answers, and (iv) Socrates asks the interlocutor to continue to search with him for a better answer. For example, in \book{Euthyphro} we find the same pattern. Socrates (i) disavows knowledge of piety, (ii) asks Euthyphro to teach him what piety is, (iii) refutes several definitions of piety by Euthyphro, but (iv) encourages Euthyphro to continue to search with him for a definition: {\g ἐξ ἀρχῆς ἄρα ἡμῖν πάλιν σκεπτέον τί ἐστι τὸ ὅσιον, ὡς ἐγὼ πρὶν ἂν μάθω ἑκὼν εἶναι οὐκ ἀποδειλιάσω} (15C11--12). The dialogue only ends when Euthyphro begs off, claiming that he needs to be somewhere else.

The procedure that Socrates follows has both destructive and constructive aspects. It's relatively clear how the elenchus can be destructive: Socrates shows that the interlocutors do now know what they claim to know by testing their definitions against their other beliefs. By doing so, he reduces them to contradiction, confusion, and incoherence.\footcite[As Gregory Vlastos has pointed out, however, there are problems about the details of the elenchus, even as a destructive technique. See Chapter 1 of][]{vlastos1991} Socrates is also eager to find the truth together with his interlocutors, but how this would work is less clear than the destructive force of the elenchus.\footcite[257]{gentzler1994}


If we view Meno's challenge from this vantage point, it looks far better than it might otherwise.\footcite[3 and 7]{nehamas1985} There's nothing eristic or lazy about asking how Socrates plans to investigate a given subject if neither he nor his interlocutors know what they are talking about. What Meno wonders is how inquiry can get off the ground or succeed if all parties to the search don't know what they're looking for.
% -]] How did we get here?

% [[- Meno's challenge
\subsection{Meno's challenge}

Meno poses his famous challenge after Socrates invites him to give yet another definition. It's difficult to say what Meno's tone is. Perhaps frustrated, but also perhaps truly confused. In any case, here is what he says:

\begin{quote}
    {\g
    Καὶ τίνα τρόπον ζητήσεις, ὦ Σώκρατες, τοῦτο ὃ μὴ οἶσθα τὸ παράπαν ὅτι ἐστίν; ποῖον γὰρ ὧν οὐκ οἶσθα πρόθεμενος ζητήσεις; ἢ εἰ καὶ ὅτι μάλιστα ἐντύχοις αὐτῷ, πῶς εἴσῃ ὅτι τοῦτό ἐστιν ὃ σὺ οὐκ ᾔδησθα;
    }

    And how will you seek something, Socrates, if you ``don't at all know''\footcite[I've quoted this following Adam Beresford's translation in][because Meno appears to be throwing back at Socrates something that Socrates said at the start of the dialogue (71B1--8)]{protagorasmeno2005} what it is? For what type of thing will you propose and search for out of all the things that you don't know? Or even if in fact you should completely happen upon it, how would you know that this is what you didn't know? (80D5--8)
\end{quote}

The challenge\footcites[Most people call it a paradox, but for the reasons given by][I think it's better to call it something else.]{moline1969}[See, however,][25--27 for support of the term `paradox']{fine2014} involves two distinct questions.\footcite[][in Chapter 7, distinguishes between these as the problems of inquiry and discovery]{scott2006}

\begin{enumerate}
    \item If you don't at all know what you're looking for, how can inquiry even start? The force of {\g προθέμενος} (80D7) is that Socrates can't even rationally determine what he's to start looking for.
    \item If you don't at all know what you're looking for, how can you be sure that any given end of inquiry has been successful? Meno grants for the sake of argument that Socrates might get lucky and fall right onto what he wanted all along—{\g ὅτι μάλιστα ἐντύχοις αὐτῷ} (80D7)—but nevertheless how will Socrates know that \textit{this} is what he was looking for?
\end{enumerate}
These are not the same problem, but they both turn on the inquirer's initial lack of knowledge. If you don't know what you're looking for, what do you aim at? And if you don't know what you're looking for, how do you know when you find it?

Although some evidence suggests that Meno is frustrated, I don't think that he's merely trying to play ``gotcha'' with Socrates. On the one hand, the rapid sequence of three questions suggests heightened emotion, and Meno appears to parrot Socrates' earlier use of {\g παράπαν} back at Socrates. In addition, Meno comes across as lazy or unintelligent to many readers.\footcites[E.g.,][60--65]{scott2006}[and the authors cited in][3]{nehamas1985} So we might worry that Meno is trying to avoid further investigation and that he's not genuinely worried about the puzzles he sets for Socrates. Although I agree that Meno is frustrated, however, he appears to have good reason to ask the questions he does and we need not take them as merely evasive. Despite Socrates' personal confidence in his methods, many interlocutors and many readers are puzzled by Socrates and his inquiries. In addition, all of his inquiries end in failure, at least in the early Platonic dialogues. None of this proves anything against him, but it does suggest that he has some explaining to do. And it suggests that Meno may be asking a legitimate and important question.

Regardless of whether Meno is serious or justified to ask what he does, Plato certainly uses the question to a serious end: it introduces the theory of recollection. Assuming the standard order of \book{Meno}, \book{Phaedo}, and \book{Phaedrus}, Plato presents this theory for the first time as part of Socrates' response to Meno's challenge. Although \book{Meno} is light on details of the theory, the prominence of the theory in that response is clear.\footnote{That is not to say that everyone agrees on the role that the theory of recollection plays in Socrates' response. In particular \smartcite{irwin1977}, \smartcite{fine1992}, \smartcite{irwin1995}, and \smartcite{scott2006} all downplay the importance of the theory of recollection in Socrates' reply to Meno.}
% -]] Meno's challenge

% [[- Socrates' restatement
\subsection{Socrates reformulates Meno's challenge}

Socrates restates Meno's challenge in an odd fashion. Although he claims to understand what Meno is getting at, his actual restatement is \textit{not} equivalent to what Meno says. Socrates argues:

\begin{enumerate}
    \item It is not possible to inquire about (i) either what a person knows or (ii) what a person does not know.
    \item In the first case, there can't be inquiry about what one knows because the person already knows it and for such a person there's no need to inquire.
    \item In the second case, the person doesn't know what they are looking for (, and so—it appears—inquiry is not possible).
\end{enumerate}
Socrates changes Meno's challenge in two ways. First, Socrates ignores Meno's question concerning the start of inquiry. Second, he adds the idea that nobody would bother inquiring after what they already know.

Both of the changes Socrates makes serve the same end. They shift the challenge from one dilemma to another. Meno poses a dilemma that inquiry can neither start nor end; Socrates responds with a dilemma that neither someone with knowledge nor without knowledge can inquire. It may make sense for Socrates to ignore questions about whether inquiry can get started. In pragmatic terms, Socratic practice seems to show that inquiries \textit{do} get started. However, this same Socratic practice also gives us reason to worry about Meno's second question since no ``What is F?'' question in the early dialogues succeeds.
% -]] Socrates' restatement

% [[- Why two versions?
\subsection{Why two versions?}

People talk a lot about whether Socrates and Meno offer the same version of the challenge to inquiry; they talk less about \textit{why} Plato offers two versions at all.
% -]] Why two versions?

% -]] Recollection in Meno
