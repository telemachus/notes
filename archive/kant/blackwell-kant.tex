% [[- LaTeX prelude
\documentclass[12pt,letterpaper]{article}

\usepackage[no-math]{fontspec}
\setmainfont{Baskerville}

\usepackage[base]{babel} % Ugh: https://tex.stackexchange.com/a/400994/29387
\usepackage[nolocalmarks]{polyglossia}
\setdefaultlanguage{english}
\setotherlanguage[variant=classic]{latin}
\setotherlanguage[variant=ancient]{greek}
\newfontfamily\greekfont[Script=Greek,Scale=MatchLowercase]{GFS Neohellenic}

\usepackage{titlesec}
\titleformat*{\section}{\large\bfseries}
\titleformat*{\subsection}{\bfseries}
\titleformat*{\subsubsection}{\bfseries}

\usepackage{parskip}
\usepackage{csquotes}
\usepackage[style=windycity,citetracker=context,backend=biber]{biblatex}
\addbibresource{kant.bib}

\usepackage{enumitem}
\setlist{noitemsep}
\usepackage[super]{nth}

\begin{hyphenrules}{latin}
    \hyphenation{}
\end{hyphenrules}

\begin{hyphenrules}{greek}
    \hyphenation{}
\end{hyphenrules}

\usepackage{fancyhdr}
\fancypagestyle{notes}{%
    \fancyhf{}
    \renewcommand{\headrulewidth}{0pt}
    \lhead{}
    \chead{\MakeUppercase{Notes on \textit{Kant} (Blackwell Great Minds)}}
    \rhead{}
    \lfoot{}
    \cfoot{\thepage}
    \rfoot{}
}
\fancypagestyle{references}{%
    \fancyhf{}
    \renewcommand{\headrulewidth}{0pt}
    \lhead{}
    \chead{\MakeUppercase{References}}
    \rhead{}
    \lfoot{}
    \cfoot{\thepage}
    \rfoot{}
}

\newcommand{\MONTH}{%
  \ifcase\the\month
  \or January% 1
  \or February% 2
  \or March% 3
  \or April% 4
  \or May% 5
  \or June% 6
  \or July% 7
  \or August% 8
  \or September% 9
  \or October% 10
  \or November% 11
  \or December% 12
  \fi}
% -]] Latex prelude

% [[- LaTeX document
\begin{document}

% [[- Title page
\begin{titlepage}
    \title{Notes on \textit{Kant} (Blackwell Great Minds)}
    \author{Peter Aronoff}
    \date{December 2020--\MONTH\ \the\year}
    \maketitle
    \thispagestyle{empty}
\end{titlepage}
% -]]

\pagestyle{notes}

% [[- Life and Works
\section*{Life and Works}

Wood sees Kant as a turning-point between early modern and contemporary philosophy.%
\footcite[Unless stated otherwise, all references come from][]{kant-wood-2005}
Looking back, we can see that Kant was still trying to ``solve the problems that had occupied philosophers in the seventeenth and eighteenth centuries'' (1).
For example, Kant still wants to provide a foundation for science, as Descartes did.
Kant also wants to understand how advances in science relate to ``traditional conceptions of metaphysics, morality, and religion'' (1).
Looking forward, however, we can see that ``Kant redefined the philosophical agenda of the early modern period, determining the problems faced by the nineteenth and twentieth centuries'' (1).
Kant changes the focus of metaphysics ``from the first-order study of the supernatural or incorporeal realm of being to the second-order study of the way human inquiry itself makes possible its access to whatever subject matter it studies'' (1).

Kant's parents, Johann Georg Kant and Anna Regina Reuter, were relatively poor.
His father was a leatherworker, and his mother's father was in the same guild.
Immanuel, the sixth of their nine children, was born on April 22, 1724.

Although Kant's parents were both pietists, Kant has mostly negative things to say about the pietist movement.
Nevertheless, Kant attended the \textit{Collegium Fredericianum} because the family's pastor got him a scholarship.
(This pastor, Franz Albert Schulz, was also the rector of the school Kant attended.)
Kant studied Latin and other subjects at the \textit{Collegium}, and he enrolled in a university at 16, in 1740.
Kant was highly critical of the dogmatism of the \textit{Collegium} and of pietism later in life.
In the same year that Kant began university, Frederick the Great became King of Prussia and recalled Christian Wolff from exile.
Thus, Kant began college as the Enlightenment found new favor in Prussia.

As a university student, Kant initially studied Latin, but then he turned to mathematics and the sciences.
From 1744 to 1755, Kant worked as a private tutor.
In 1755, he began to teach at the university once he had achieved advanced degrees.
Kant wrote primarily on the natural sciences early in his career, but his second dissertation in 1755 was more philosophical.
In it, Kant discussed the application of the principle of sufficient reason within metaphysics.
The job Kant received in 1755 did not pay him a direct salary.
Kant could offer lectures, which students paid him for, but the university paid him no salary.
Thus, Kant had to offer popular lectures if he wanted to earn a living.
Although Kant taught and wrote about metaphysics, he lectured a great deal on the sciences and published on those questions as well.

Wood denies several clichés about Kant and his life.
First, Wood does not believe that Hume woke Kant up from ``a dogmatic slumber.''
Although Kant himself said that Hume played this role for him, Wood thinks that this is just a rhetorical move of Kant's.
Instead, Wood believes that Kant was critical of Wolffian metaphysics as early as 1755.
Second, people often say that Kant was rigid, inhuman, and robotic in his habits and tastes.
Wood argues that Kant was gregarious and social as a young man and that biographers exaggerate Kant's rigidity even when he was older.
(Wood does allow, however, that Kant became more regular in his habits under the influence of a close friend, Joseph Green.)

During the 1770s, Kant published very little, but in 1781, the first edition of his \textit{Critique of Pure Reason} appeared.
Wood stresses that Kant was working on \textit{CPR} for nearly a decade although the actual writing was relatively brief---four months in 1779 and 1780.
Kant was unhappy that the book was mostly ignored or misunderstood.
Although he wrote \textit{Prolegomena to any Future Metaphysics} in order to reach a wider audience, Kant did not succeed in making \textit{CPR} or his ideas more popular.
According to Wood, Christoph Wieland helped Kant reach a wider audience by writing a series of articles about Kant in 1786.
Kant published a second edition of \textit{CPR} in 1787 in order to clarify his ideas and reply to some critics.
Kant wrote his \textit{Critique of Practical Reason} as part of this revision, but he decided to publish it separately because the book became too long.

According to Wood, Kant never planned to write three critiques on the three topics of metaphysics, morality, and art.
The second and third critiques developed organically from work Kant did subsequent to the first critique.
In particular, Wood connects the second critique to Kant's writings about history, the Englightenment, and politics.
Wood also stresses that Kant only gave his final views about ethics in his \textit{Metaphysics of Morals} in 1797.
For this reason, Wood believes that we should read Kant's \textit{Groundwork for the Metaphysics of Morals} in the light of the \textit{Metaphysics of Morals}.
(This worries me since the \textit{Groundwork} appeared in 1785, twelve years before the \textit{Metaphysics of Morals} began to appear.%
\footnote{Wood cites \textit{Metaphysics of Morals} as 1797--1798, so I assume it appeared over two years.}%
)

Kant spent much of the 1790s dealing with various disputes and external pressures.
On the one hand, now that he had become famous, he had a lot of criticism and misunderstanding to respond to.
On the other hand, a new king began to restrict what scholars could publish about religion.
Kant cleverly avoided censors when he published \textit{Religion Within the Boundaries of Mere Reason} (1792).
He gave the book to an academic body (at the University of Jena) which was a legal way to get around governmental censorship.
Nevertheless, the king wrote Kant telling him not to write about religion again, and Kant agreed to this ban.
Wood has a good discussion here of Kant's politics and morality (18--19).
In brief, Kant felt that political subjects should obey even unjust commands from a legitimate authority except when the command was to do something positively immoral or evil.

Kant retired in 1796, and he died in 1804 when he was just shy of eighty years old.
In his final years, Kant had three main goals.
Kant wanted to finish his \textit{Metaphysics of Morals}, and he did so in 1798.
Kant wanted to publish more of his lectures.
In this case, he published his anthropology lectures in 1798, but left the rest for others to do.
Kant also began an ambitious ``new work centering on the transition between transcendental philosophy and empirical science'' (22).
Kant left this last project unfinished, but the fragments were later published as his \textit{Opus Postumum}.
% -]] Life and Works

% [[- Bibliography
\newpage
\pagestyle{references}
\defbibfilter{primary}{%
    keyword=primary
}
\defbibfilter{secondary}{%
    keyword=secondary
}

% \nocite{*}
\printbibliography[filter=primary,title={Primary Sources}]
\printbibliography[filter=secondary,title={Secondary Sources}]
% -]] Bibliography

\end{document}
% -]]
