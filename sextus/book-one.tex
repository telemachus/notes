% [[- Chapter title
\chapter{Book I}
% -]] Chapter title

% [[- The Most General Difference of Philosophies (1--4)
\section*{The Most General Difference of Philosophies (1--4)}


Sextus begins thus:

\begin{quote}
    When people inquire into any matter, the likely result is either (i) a discovery, or (ii) a denial of discovery and an agreement of inapprehensibility, or (iii) a continuation of inquiry.
\end{quote}

In other words, if someone investigates, say, if \textit{p}, then we can expect them to find out, to fail to find out and say that it is not possible to find out, or to keep investigating.

Sextus uses this to distinguish between three types of philosopher. Those properly called dogmatists think that they have discovered the truth. Academics assert that it is not possible to discover the truth. They are therefore a kind of negative dogmatist. The skeptics continue to search for answers.

Sextus will speak about the skeptics, and even there only with provisos. He disavows any attempt to affirm strongly how things actually are. Instead, he will ``report like a storyteller about each matter in accordance with how things current seem to him.'' \textbf{Nota bene}: I wonder what we would find if we investigated in detail how other writers use the adverb \textgreek{ἱστορικῶς}, which Sextus uses to qualify \textgreek{ἀπαγγέλλομεν} here.
% -]] The Most General Difference of Philosophies (1--4)

% [[- The Accounts of Skepticism (5--6)
\section*{The Accounts of Skepticism (5--6)}


Sextus subdivides skeptical philosophy into a general and a specific part. He explains that the general part of skepticism concerns the character of skepticism itself. In the general part, we consider questions about the intention, starting points, criterion, goals of skepticism, skeptical assertions, and the differences between skepticism and nearby schools of thought. In the specific part, the skeptic argues against each part of what other people call philosophy. I think this alludes to the division of philosophy into logic, physics, and ethics.
% -]] The Accounts of Skepticism (5--6)

% [[- The Names Applied to Skepticism (7)
\section*{The Names Applied to Skepticism (7)}


Sextus considers several names that people use for \textgreek{ἡ σκεπτικὰ ἀγωγή}: zetetic, ephetic, aporetic, and Pyrrhonian. People call skeptics zetetic because they inquire and investigate so conspicuously. People call skeptics ephetic because of the experience they undergo after inquiry. That is, people call skeptics ephetic because they withold judgment or suspend judgment. People call them aporetic either because they are at a loss about everything and inquire or because they they are at a loss about assent and denial. That is, Sextus appears to distinguish here between the beginning and process of skeptical inquiry, on the one hand, and the failure to conclude this inquiry, on the other hand. Finally, people call skeptics Pyrrhonian because ``Pyrrho appears to us to have advanced in skepticism more bodily and conspicuously than anyone before him.''

Scholars have taken Sextus to be cautious or guarded in his comments about Pyrrho, but I am not so sure. Sextus certainly doesn't call Pyrrho the first skeptic, but I think that what he \textit{does} say is perfectly full throated.

However, I am unsure how to interpret \textgreek{σωματικώτερον} here. LSJ doesn't have a lot on the adverbial use of this adjective. They list a literally use (``corporeally''), a use in astrology where it is opposed to ``in aspect'' (huh?), and a legal context where punishment can be taken ``in money or physically.'' I suppose that the sense is that Pyrrho committed himself to skepticism ``body and soul,'' as we might say.
% -]] The Names Applied to Skepticism (7)

% [[- What is Skepticism? (8--10)
\section*{What is Skepticism? (8--10)}

Sextus tells us that skepticism is something one does rather than a body of ideas. This only makes sense since a Pyrrhonian skeptic has no characteristic body of doctrine. (See Frede and Striker on this.) More specifically, Sextus calls skepticism ``an ability to place appearances and thoughts in conflict in any way at all, from which ability skeptics arrive first at suspension of judgment and then at ataraxia as a result of the equal strength of opposed matters and arguments.''

Sextus then goes through each part of this definition giving us further guidance on how to take his words. In many cases, he specifies how to take his words. For example, by \textgreek{δύναμις} he means simply ability, and by \textgreek{φαινόμενα} he means perceptual appearances since he opposes them (here) to \textgreek{νοητά}. However, in other cases, Sextus offers more than one interpretation. For example, he suggests that we can join \textgreek{καθ᾽οἱονδήποτε τρόπον} with \textgreek{δύναμις} or with \textgreek{ἀντιθετικὴ φαινομένων τε καὶ νοουμένων}. I find the effect of this jarring, but it is a feature of Sextus's writing. He often tells us that things are ambiguous, and he does not much care to settle those ambiguities.

Sextus offers a third interpretation of \textgreek{καθ᾽οἱονδήποτε τρόπον} which I find most interesting of all. We can take this phrase closely with \textgreek{φαινομένων τε καὶ νοουμένων} insofar as Sextus does not care \textit{how} things that appear or are thought appear or are thought. As he says ``we simply take these.'' The point seems to be that, for the moment at least, Sextus is willing to take these as givens which he does not care to investigate.
% -]] What is Skepticism? (8--10)

% [[- The Skeptic (11)
\section*{The Skeptic (11)}

Sextus is very brief on this topic: a Pyrrhonian skeptic is pretty much just someone who has the ability outlined in iii.
% -]] The Skeptic (11)

% [[- The Sources of Skepticism (12)
\section*{The Sources of Skepticism (12)}

The causal source of skepticism is the hope of \textgreek{ἀταραξία}. To explain, Sextus tells a story. Very intelligent people were disturbed when they noticed the anomaly in things, and they were at a loss as to which view about things they should assent to. So they began to investigate. Their intention was to decide how things are and thus free themselves from this disturbance. This is the initiating cause of skepticism.

Sextus adds that the cause of the skeptical party or constitution is primarily the claim that there is an equal argument for every argument. Sextus does not say it, but by this he seems to mean that \textgreek{isosthenia} brings the skeptics together as a group. What he does say is that \textgreek{isosthenia} leads them to cease to dogmatize. In a way, this too \textit{is} what defines them as a group. (See i.)
% -]] The Sources of Skepticism (12)

% [[- Does the skeptic have beliefs? (13--15)
\section*{Does the skeptic have beliefs? (13--15)}

Sextus begins by telling us a sense of ``have beliefs'' (\textgreek{δογματίζειν}) in which he does \textit{not} say that the skeptic does not have beliefs. This is odd, and makes things complicated right away. There is a sense in which to have beliefs is simply to agree to some matter. Sextus does not say that skeptics do not do this since, as he says, ``the skeptic assents to experiences which are perceptually forced on them'' (\textgreek{τοῖς γὰρ κατὰ φαντασίαν κατηναγκασμένοις πάθεσι συγκατατίθεται ὁ σκεπτικός}). He then gives two examples: if a skeptic is warmed or cooled, the skeptic will not say ``I am not warmed or cooled.'' Notice that again this is all multiply negative. 

Sextus then considers a way in which the skeptic will not have beliefs. The skeptic will not have beliefs if by ``have beliefs'' we mean ``assent to some unclear matter investigated by the sciences.'' In fact, Sextus continues, the skeptic will not have beliefs in this sense since the skeptic will not assent to \textit{anything} unclear.

According to Sextus, when the skeptic utters the skeptical phrases such as ``no more,'' the skeptic also does not hold beliefs. A person who holds beliefs posits as real (or ``as really existing'' or ``as true,'' \textgreek{ὡς ὑπάρχον}) what they have beliefs about, but the skeptic does not take this attitude towards the skeptical phrases. The skeptical phrases apply to themselves as much as to other things. Hence, the skeptic does not think of them as altogether real or true. Most importantly, the skeptic can utter their special phrases and not hold beliefs because the skeptic only says what appears to them and reports their (passive) experience without belief and without affirming anything about underlying external realities.

Sextus leaves a great deal unsaid here. In no particular order, (i) Sextus does not clarify the relationship between the beliefs that a skeptic may have and ordinary belief, (ii) Sextus does not explicitly state whether a skeptic can have non-dogmatic beliefs about the subject matter of the sciences or philosophy, (iii) Sextus does not say whether skeptical beliefs are limited by subject, strength, formation, or anything else. He lists a few things that a skeptic may not do (``assent to anything unclear'' and ``posit as real'' what they talk about), but this still leaves a great many questions.
% -]] Does the skeptic have beliefs? (13--15)

% [[- Does the Skeptic Belong to a School?
\section*{Does the Skeptic Belong to a School?}

Whether a skeptic belongs to a school depends on what you mean by ``belong to a school.'' If you mean that someone who belongs to a school accepts many beliefs that (i) are internally coherent, (ii) fit with appearances, and (3) involve assent to unclear matters, then the skeptic does not belong to a school. However, the skeptic does belong to a school in the sense that skeptics have (i) a way of life that (ii) follows some rationale (iii) in accordance with appearances; and (iv) that rationale indicates how it is possible to live in an apparently correct manner. Sextus clarifies that by ``a correct manner'' he means something more simple than ``in accordance with virtue.'' And whatever he means includes the possibility of suspension of judgment.
% -]] Does the Skeptic Belong to a School?
