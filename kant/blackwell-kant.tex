% [[- LaTeX prelude
\documentclass[12pt,letterpaper]{article}

\usepackage[no-math]{fontspec}
\setmainfont{Baskerville}

\usepackage[base]{babel} % Ugh: https://tex.stackexchange.com/a/400994/29387
\usepackage[nolocalmarks]{polyglossia}
\setdefaultlanguage{english}
\setotherlanguage[variant=classic]{latin}
\setotherlanguage[variant=ancient]{greek}
\newfontfamily\greekfont[Script=Greek,Scale=MatchLowercase]{GFS Neohellenic}

\usepackage{titlesec}
\titleformat*{\section}{\large\bfseries}
\titleformat*{\subsection}{\bfseries}
\titleformat*{\subsubsection}{\bfseries}

\usepackage{parskip}
\usepackage{csquotes}
\usepackage[style=windycity,citetracker=context,backend=biber]{biblatex}
\addbibresource{kant.bib}

\usepackage{enumitem}
\setlist{noitemsep}
\usepackage[super]{nth}

\begin{hyphenrules}{latin}
    \hyphenation{}
\end{hyphenrules}

\begin{hyphenrules}{greek}
    \hyphenation{}
\end{hyphenrules}

\usepackage{fancyhdr}
\fancypagestyle{notes}{%
    \fancyhf{}
    \renewcommand{\headrulewidth}{0pt}
    \lhead{}
    \chead{\MakeUppercase{Notes on \textit{Kant} (Blackwell Great Minds)}}
    \rhead{}
    \lfoot{}
    \cfoot{\thepage}
    \rfoot{}
}
\fancypagestyle{references}{%
    \fancyhf{}
    \renewcommand{\headrulewidth}{0pt}
    \lhead{}
    \chead{\MakeUppercase{References}}
    \rhead{}
    \lfoot{}
    \cfoot{\thepage}
    \rfoot{}
}

\newcommand{\MONTH}{%
  \ifcase\the\month
  \or January% 1
  \or February% 2
  \or March% 3
  \or April% 4
  \or May% 5
  \or June% 6
  \or July% 7
  \or August% 8
  \or September% 9
  \or October% 10
  \or November% 11
  \or December% 12
  \fi}
% -]] Latex prelude

% [[- LaTeX document
\begin{document}

% [[- Title page
\begin{titlepage}
    \title{Notes on \textit{Kant} (Blackwell Great Minds)}
    \author{Peter Aronoff}
    \date{December 2020--\MONTH\ \the\year}
    \maketitle
    \thispagestyle{empty}
\end{titlepage}
% -]]

\pagestyle{notes}

% [[- Life and Works
\section*{Life and Works}

Wood sees Kant as a turning-point between early modern and contemporary philosophy.%
\footcite[Unless stated otherwise, all references come from][]{kant-wood-2005}
Looking back, we can see that Kant was still trying to ``solve the problems that had occupied philosophers in the seventeenth and eighteenth centuries'' (1).
For example, Kant still wants to provide a foundation for science, as Descartes did.
Kant also wants to understand how advances in science relate to ``traditional conceptions of metaphysics, morality, and religion'' (1).
Looking forward, however, we can see that ``Kant redefined the philosophical agenda of the early modern period, determining the problems faced by the nineteenth and twentieth centuries'' (1).
Kant changes the focus of metaphysics ``from the first-order study of the supernatural or incorporeal realm of being to the second-order study of the way human inquiry itself makes possible its access to whatever subject matter it studies'' (1).

Kant was born in 1724 to relatively poor parents.
His father was a leatherworker, and his mother's father was in the same guild.
Although Kant's parents were both pietists, Kant has mostly negative things to say about the pietist movement.
Nevertheless, Kant attended the \textit{Collegium Fredericianum} because the family's pastor got him a scholarship.
(This pastor, Franz Albert Schulz, was the rector of the school Kant attended.)
Kant studied Latin and other subjects at the \textit{Collegium}, and he enrolled in a university at 16, in 1740.
In the same year, Frederick the Great became King of Prussia, and he recalled Christian Wolff from exile.
Thus, Kant began college as the Enlightenment found new favor in Prussia.

Kant spent his entire life in Königsberg.
% -]] Life and Works

% [[- Bibliography
\newpage
\pagestyle{references}
\defbibfilter{primary}{%
    keyword=primary
}
\defbibfilter{secondary}{%
    keyword=secondary
}

% \nocite{*}
\printbibliography[filter=primary,title={Primary Sources}]
\printbibliography[filter=secondary,title={Secondary Sources}]
% -]] Bibliography

\end{document}
% -]]
