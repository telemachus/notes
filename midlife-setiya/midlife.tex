% [[- Prelude
\documentclass[12pt,letterpaper]{article}

\usepackage[no-math]{fontspec}
\setmainfont{Baskerville}

\usepackage[base]{babel} % Ugh: https://tex.stackexchange.com/a/400994/29387
\usepackage[nolocalmarks]{polyglossia}
\setdefaultlanguage{english}
\setotherlanguage[variant=classic]{latin}
\setotherlanguage[variant=ancient]{greek}
\newfontfamily\greekfont[Script=Greek,Scale=MatchLowercase]{GFS Neohellenic}

\usepackage{titlesec}
\titleformat*{\section}{\large\bfseries}
\titleformat*{\subsection}{\bfseries}
\titleformat*{\subsubsection}{\bfseries}

\usepackage{parskip}
\usepackage{csquotes}
\usepackage[style=windycity,citetracker=context,backend=biber]{biblatex}
\addbibresource{midlife.bib}

\usepackage{enumitem}
\setlist{noitemsep}
\usepackage[super]{nth}

\begin{hyphenrules}{latin}
    \hyphenation{}
\end{hyphenrules}

\begin{hyphenrules}{greek}
    \hyphenation{}
\end{hyphenrules}

\usepackage{fancyhdr}
\fancypagestyle{notes}{%
    \fancyhf{}
    \renewcommand{\headrulewidth}{0pt}
    \lhead{}
    \chead{\MakeUppercase{Notes on \textit{Midlife: A Philosophical Guide}}}
    \rhead{}
    \lfoot{}
    \cfoot{\thepage}
    \rfoot{}
}
\fancypagestyle{references}{%
    \fancyhf{}
    \renewcommand{\headrulewidth}{0pt}
    \lhead{}
    \chead{\MakeUppercase{References}}
    \rhead{}
    \lfoot{}
    \cfoot{\thepage}
    \rfoot{}
}

\newcommand{\MONTH}{%
  \ifcase\the\month
  \or January% 1
  \or February% 2
  \or March% 3
  \or April% 4
  \or May% 5
  \or June% 6
  \or July% 7
  \or August% 8
  \or September% 9
  \or October% 10
  \or November% 11
  \or December% 12
  \fi}
% -]] Prelude

% [[- Document
\begin{document}

% [[- Title
\begin{titlepage}
    \title{Notes on \textit{Midlife}}
    \author{Peter Aronoff}
    \date{January 2021--\MONTH\ \the\year}
    \maketitle
    \thispagestyle{empty}
\end{titlepage}
% -]] Title

\pagestyle{notes}

% [[- Chapter 1: A Brief History of the Midlife Crisis
\section*{Chapter 1: A Brief History of the Midlife Crisis}

In this first chapter, Setiya gives a brief history of the concept of a midlife crisis.%
\footcite[Unless I say otherwise, all citations refer to][]{midlife-setiya-2017}
Setiya discusses two concepts of midlife, and he stresses that they differ in important ways.
The first concept originates in psychoanalysis, and the second in economics.
I suspect that most people think of the psychoanalytic concept when they hear or read \textit{midlife crisis}.
As a result, we should carefully distinguish them if possible.
% -]] Chapter 1: A Brief History of the Midlife Crisis

% [[- A Psychoanalytic Midlife Crisis
\subsection*{A Psychoanalytic Midlife Crisis}

The concept of a midlife crisis springs from psychoanalysis.
In 1965, Elliott Jaques publishes ``Death and the mid-life crisis'' in \textit{International Journal of Psychoanalysis}.
Although Jaques does not invent the concept out of nothing, he gives midlife crisis its name and launches its career.
Following Jacques, other psychologists and psychiatrists examine midlife through studies and interviews, and in 1976 Gail Sheehy, a journalist, popularizes the concept in \textit{Passages: Predictable Crises of Adult Life}.

We get the popular understanding of a midlife crisis from these works and others like them.
According to this understanding, when people reach midlife---their thirties or forties---they experience a physical, emotional, and social crisis.
At midlife, physical strength and health decline, progress at work stalls or becomes more difficult, and family life seems more constricting than rewarding.
People feel that new and good experiences are all in the past: the future will only be the same or worse.
In addition, people experience a crisis of meaning and purpose.
They lose sense of why they do what they do; they lose faith that anything is genuinely worth doing.

Setiya stresses that the popular conception exaggerates or mistakes key features of the psychoanalytic midlife crisis.
The popular conception sometimes claims that a midlife crisis is a universal feature of human life, but Jaques and company did not claim this.
The popular conception usually sees a midlife crisis as almost entirely negative, but again early writers discuss rebirth and renewal as much as anxiety and tedium.
Nevertheless, the universal and negative conception is widespread.

Another problem for the psychoanalytic midlife crisis is replication.
In the 1990s a very large study ``revealed a positive portrayal of aging'' (14).
In another study, only 26 percent of respondents said they experienced a midlife crisis.
Further work suggests that even that number is too high since people used a very loose definition of ``midlife crisis.''
That is, some said they had a midlife crisis if anything difficult happened to them in their thirties and forties.
Further studies have supported the reversal, and many social scientists now view the psychoanalytic midlife crisis as a myth, exaggeration, or popular misunderstanding.
% -]] A Psychoanalytic Midlife Crisis

% [[- An Economic Midlife Crisis
\subsection*{An Economic Midlife Crisis}

Nevertheless, economists have given the midlife crisis a second life.
In the 1980s and 1990s, economists began to study well-being in addition to money.
In that context, two economists argued in 2008 that well-being shows a U-shaped curve over a person's life.
They consider well-being starting from young adulthood, and they argue that it begins high, bottoms out in the mid-forties, and then climbs again as people grow older.
They claim that this finding applies cross-culturally to men and women, and subsequent studies confirm their findings.
Setiya also mentions a study from 2012 that finds a U-shaped curve in the well-being of great apes (20).

How is this concept of a midlife crisis different from the earlier one?
The earlier concept stresses crisis as catastrophe, and it suggests that things continue downhill after midlife.
This new concept only posits a relative low point, and it acknowledges that many people have a gentler midlife crisis.
Perhaps most importantly, the economic midlife predicts an upturn in well-being after the relative low point.
In all these ways, economics paints a more positive picture than psychoanalysis.

Setiya endorses Hannes Schwandt, who explains the U-shaped curve in terms of desires and expectations.
In Setiya's description,
``Schwandt learned that younger people tend to over-estimate how satisfied they will be, while mid-lifers underestimate old age.
Middle age is consequently worse than anticipated and at the same time hopes for the future fade.
Hence the dismal vertex of the U-curve'' (21).

Although the economic midlife is relatively positive, Setiya argues that the U-curve ``resonates'' with the psychoanalytic crisis (20--21, quotation on 21).
If the U-curve is the relative low point for most people, then \textit{that} is where we should expect to see a majority of crises.
According to Setiya, some economic studies bear this out: the greatest likelihood of a traumatic midlife crisis is at the bottom of the U-curve.
In addition, Schwandt explains the U-curve in a way that echoes psychoanalysis.
At the bottom of the U-curve, according to Schwandt, people anticipate a poor future and believe that they have not achieved all they should have.

Although Setiya acknowledges that some scholars deny even the U-curve, he will employ that conception of midlife in this book.
Setiya also notes that this book is personal in two ways.
First, Setiya himself writes the book partly because he has experienced some form of midlife crisis.
He has asked, ``Chapter 2: Is That All There Is?''
Second, Setiya will approach the questions that the U-curve poses as the philosopher he is rather than as a psychologist or economist.
In the final section of this chapter, he explains what a philosophical approach means.
% -]] An Economic Midlife Crisis

% [[- A Philosophical Approach
\subsection*{A Philosophical Approach}

On the one hand, we would expect philosophers to study midlife crises.
During a midlife crisis, some people experience a depressing sense that nothing has any value.
Philosophers have quite a lot to say about value---what matters and why.
The U-curve also tracks well-being over the course of a life, and philosophers certainly care about well-being.

Nevertheless, philosophers have not paid much attention to midlife, and Setiya thinks that there are some good reasons why they have not.
First, philosophy often avoids ``the grimy facts of aging and bodily decay'' (24).
Second, people predominantly investigate midlife via empirical studies, and philosophy doesn't do empirical studies.%
\footnote{We might quibble: there is such a thing as experimental philosophy.
Nevertheless, I think Setiya makes a reasonable point.}
Even when philosophers contribute to empirical research, they do so at a very high level of abstraction.
For example, philosophers have views about counter-factual reasoning, but these views are very general.
Thus, even here, philosophers don't contribute specifically to the study of midlife.
Third, Setiya believes that ``philosophers look for questions that feel  timeless and universal'' (24).
For example, Setiya refers to Kant and Aristotle.
Kant examines the question ``What should I do?''
Aristotle investigates what makes a good life good.
Neither Kant nor Aristotle seem to leave room for the specific challenges of midlife.

Nevertheless, Setiya thinks that midlife raises issues that people must face and that philosophers should consider.
At midlife, Setiya notes, a person ``has a meaningful past and a meaningful future, both of which [they] must confront'' (25--26)
As Setiya points out, however, midlife involves a more complex temporal situation than Aristotle or Kant imagine.
When Aristotle evaluates human flourishing, he looks back on completed lives from the outside.
When Kant asks what we should do, he looks exclusively towards the future, which he appears to imagine as completely open.
In midlife, however, we look backward and forward.
Unlike Aristotle, at midlife we can only consider part of a completed life: we cannot yet see the whole that Solon recommends.
We can begin to judge our lives, but we cannot render a final judgment.
Unlike Kant, at midlife we do not ask what we should do as if everything is on the table and nothing has already happened.
At midlife, we have both a past and a future.
We have made many choices already, and past choices can make future choices more difficult or impossible.

How does Setiya understand the specifically \textit{philosophical} nature of his book?
According to Setiya, ``the methods of philosophy are reflection and reasoning'' (27).
He explicitly disavows the methods of social science: he will present no experimental results, and he doesn't expect us to take his word for things.
He will give us ``analyses and arguments'' (27), and we should only accept what we ``can confirm [ourselves]'' (28).

In addition, Setiya describes the book as ``a form of cognitive therapy'' (26).
Although this may not sound abstract enough for philosophy, Setiya appears to have in mind something like Stoic or Epicurean emotional therapy.
Setiya can improve how we feel and how we behave by discussing ``abstract questions of knowledge, value, and time'' (27).
As our beliefs change, so do our emotions and our actions.
Hence, the therapy truly is cognitive.

At the same time, Setiya says that his ``pursuit of these ideas is personal and introspective'' (26).
Again, he will not rely on laboratory studies or social science for data.
Instead, he will consider his own experiences, as well as those of novelists, philosophers, and other figures.
Although Setiya examines accounts from different people, he avoids questions about the ``social construction'' (27) of midlife.
He also avoids some familiar clichés: he doesn't say much about ``fast cars and wild affairs'' (27).
% -]] A Philosophical Approach

% [[- Chapter 2: Is That All There Is?
\section*{Chapter 2: Is That All There Is?}

In Chapter 2, Setiya considers John Stuart Mill.
In some ways, Mill does not provide an appropriate example, as Setiya acknowledges.
First, Mill had an extremely idiosyncratic childhood, so we can't expect other people to have the same problems as him for the same reasons.
Second, Mill experienced his crisis early in life, at the age of twenty.
Nevertheless, Setiya argues that Mill can help us understand midlife.
First, Mill undergoes a breakdown that partially echoes the midlife crisis.
Second, Setiya will use this chapter to discuss some key distinctions.
% -]] Chapter 2: Is That All There Is?

% [[- Mill's Breakdown
\subsection*{Mill's Breakdown}

At the age of twenty, Mill had a breakdown.
Mill's father raised him in an experimental and relentless manner.
James Mill, under the influence of Jeremy Bentham, attempted to make John Stuart Mill a revolutionary figure, someone who would grow up to make the world better.
In practice, this meant that Mill had very limited exposure to other children or people.
Mill learned ancient Greek starting at 3, he began Latin at 8, and in his teens he studied economics, psychology, social theory, and the law.
By the time of his breakdown, Mill had already received the equivalent of graduate training in multiple disciplines.
At the same time, Mill had nearly nothing that we would recognize as a childhood or a social life.

In his \textit{Autobiography}, Mill presents the breakdown as a kind of depression.
He writes that things that normally gave him pleasure seemed ``insipid or indifferent.''%
\footcite[77]{mill-2018}
% -]] Mill's Breakdown

% [[- Bibliography
\newpage
\pagestyle{references}
% \defbibfilter{primary}{%
%     keyword=primary
% }
% \defbibfilter{secondary}{%
%     keyword=secondary
% }

% \nocite{*}
% \printbibliography[filter=primary,title={Primary Sources}]
% \printbibliography[filter=secondary,title={Secondary Sources}]
\printbibliography[title={References}]
% -]] Bibliography

\end{document}
% -]] Document
