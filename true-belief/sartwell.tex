% [[- Section: Sartwell
\section{Sartwell}

% -]] Section: Sartwell

% [[- Knowledge is merely true belief
Crispin Sartwell offers the following analysis of \textit{S} knows that \textit{p} \parencite[157]{sartwell1991}:

\begin{enumerate}
    \item \textit{p} is true
    \item \textit{S} believes that \textit{p}.
\end{enumerate}

He knows that this view will be controversial, and this first article is largely devoted to shifting the burden of proof rather than proving his new analysis.
% -]] Knowledge is merely true belief

% [[- counter-examples
Sartwell considers several possible counter-examples \parencite[158--159]{sartwell1991}:

\begin{enumerate}
    \item Someone believes that the Pythagorean theorem is true, on the basis of a dream.
    \item Someone bets on a horse to win a race after choosing the horse's name at random. The horse wins the race.
    \item Someone's brain is operated on, and as a result that person is ``disposed to assent'' to claims that Goldbach's conjecture is true. Stipulate that Goldbach's conjecture is true.
    \item Someone comes to believe something after reading tea leaves or astrology. The belief is true
\end{enumerate}

Sartwell takes two approaches to these counter-examples. Some of them, he is inclined to dismiss on the grounds that they do not meet the requirements of his analysis. Others, he's willing to accept as examples of knowledge, however paradoxical this might seem. I'll consider each type of reply in turn.
% -]] counter-example

% [[- Not belief
Sartwell replies to the third and the second case by denying that they fit his analysis. In both cases, he focuses on the requirement that the person must \textit{believe} the claim in question. Sartwell objects that attributions of belief require sufficient commitment as well as adequate understanding.

First, let's consider the gambling example. Few of us would be inclined to say that the gambler actually believes that the horse will win the race. Sartwell remarks that the person lacks ``sufficient commitment''. The person may act in ways that suggest belief (making the bet, for example), but we can attribute these actions to other causes such as compulsion.

In the case of the Goldbach conjecture, the problem is that many beliefs cannot exist in isolation. The original example is rather underdeveloped, intentionally as it turns out. The mere disposition to assent to a proposition cannot be enough to show that we believe it. In order to believe something about Goldbach's conjecture, we need to understand it. And in order to understand it, we must have a host of other beliefs (and perhaps knowledge), and these various beliefs (and possibly knowledge) must be structured in a specific way. In short, assent is not sufficient for belief.

Finally, Sartwell sums up by reminding us that many writers will argue for a ``third requirement'' on knowledge, over and above belief and truth, in a very slipshod way. Some arguments are effectively this: a lucky guess is not knowledge. But a lucky guess, in most cases, isn't even a belief.
% -]] Not belief

% [[- Bite the bullet
Sartwell does not explicitly consider the other cases right away, but based on what he says in the rest of the article, I think he's inclined to accept them as knowledge. In the later part of the essay, he argues that justification and rationality are important to epistemology, but for instrumental reasons only. Well-justified beliefs formed by rational methods are more likely to be true, and truth is required for knowledge. So it makes perfect sense that we care about justification. As Sartwell puts it, justification is a criterion of knowledge. But, he also writes, it is not a logically necessary condition of knowledge.
% -]] Bite the bullet
